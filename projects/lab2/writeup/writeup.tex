\documentclass{article}

\usepackage[margin=1in]{geometry}
\usepackage{graphicx}
\usepackage{amsmath}

\setlength{\parindent}{0pt}

\begin{document}

\noindent 18-748 Lab 2 \hfill Group 17

\noindent \hrulefill

\section*{Architecture}

Master runs a state machine with the following states:
\begin{enumerate}
\item idle (entered upon reset)
\item waiting for user to hit a remote mole
\end{enumerate}

Transitions are:
\begin{enumerate}
\item upon reset: randomly picks a mole slave node (excluding the previously
picked one), sends a message of type 'arm', and transitions from idle
state to 'waiting for user' state
\item in 'waiting for user' state: sends message of type 'request status'
to the chosen node, upon getting 'status type' reply state remains unchanged,
but upon getting a 'scored' type reply, state transitions from 'waiting for
user' to idle state
\item upon timeout expiration master sends a 'reset' message to the mole node
\end{enumerate}

Each slave runs a state machine with the follwing states:
\begin{enumerate}
\item idle (entered upon reset)
\item armed (mole is awake and awaiting to be hit by user)
\item scored (mole hit but the fact was not yet communicated to master)
\end{enumerate}

The transitions are:
\begin{enumerate}
\item upon receiving 'arm' message, transition from idle state to armed state
\item upon detecting light intensity change in 'armed' state, transition
to scored state
\item upon receiving status message in 'scored' state, reply with message
of type 'stored' and transition to idle state
\item upon receiving status message in other states, reply with current state
\end{enumerate}

Invariant: the slave never initiates communication: all its packates are replies
to packets from the master node.

Our system does not implement acknowledgements for simplicity.
Master ignores lost replies to its status requests, but
re-issues them to the same node indefinitely.

Our system reduces timeouts based on the number of the round
in order to challenge the player.

\section*{Latency Estimate}

The latency comes from the following sources:
\begin{enumerate}
\item operating system and network stack overhead on sender node
\item MAC access delay (due to contention)
\item Flight of signal (negligible)
\item Delay on receiver due to \textbf{low-power listening} feature of BMAC

The BMAC LPL protocol implies that the receiving node will only check for
packets at a preset interval, which in our case is configured to be 100 ms in
\texttt{bmac.h} as \verb!BMAC_DEFAULT_CHECK_RATE!. We conclude that this source
of delay is the most significant.  So, until a remote mole is awake, one might
have to wait more than 100ms.
\end{enumerate}

\section*{Battery Lifetime Estimate}

The master node polls the designated mole slave node (and only that node)
at a fixed interval of 1 second. Each 'poll' is an exchange of exactly
two packets (request packet and response packet). Each packet payload is below
4 bytes (node id and message type). Estimate header and checksum overhead
at two bytes. Overall conservative  estimate is 6 bytes. So, the data
amount to transmit is about 6*2*8=96 bits per second, round to 128 bps.
Radio bandwidth is ~256 Kbps in burst rate. So, the data transmission
can be performed in about 128/256k = 500 us, round to 1 ms. This means
a duty cycle of 1 ms of transmission every second, i.e. 0.1\%. Current during
active transmission is 20 mA (see class slide on RF230), which at 100\% duty
cycle would drain two 2300 mAh AA batteries in 2300 * 2 mAh / 20 mA = 460 h.
But, at 0.1\% duty cycle this is 460 000 h. From this needs to be
subtracted the idle power consumption (72uA rounded to ~0.1uA), which
would drain the battery in about 46,000 h. But even that is not the
bottleneck.  From this needs to be subtracted the compute power for the
microcontroler (~10mA). In our implementation we do not release the processor,
therefore tihs will be consumed at 100\% duty cycle. Therefore the battery
will be drained faster than in 460 h = 19 days.



\end{document}
