\documentclass{article}

\usepackage[margin=1in]{geometry}
\usepackage{graphicx}
\usepackage{amsmath}

\setlength{\parindent}{0pt}

\begin{document}

\noindent 18-748 Lab 3 \hfill Group 17

\noindent \hrulefill

\section{Design}

Our network supports routed messages between any two nodes.

The routing is implemented within the \texttt{router} task and the
\texttt{rx\_task}. The \texttt{rx\_task} is responsible for always listening
and adding received packets to a queue without fully parsing them. The
\texttt{router} task waits until this queue get an element and processes the
packets.

The API into our router consists of
\begin{itemize}
\item \texttt{send\_message(recipient, payload, length)} function, and
\item \texttt{receive\_messages(sender, payload, length)} callback.
\end{itemize}

The \texttt{send\_message} is designed to be invoked from tasks other than the
\texttt{router}, and enqueues the messages onto a queue, which is
asynchronously processed by the router task. This is important because the
router is not allowed to synchronously wait for acks.

The routing mechanism is routing tables on each node with next hop information
for each other node in the network. This is not scalable, but sufficient for
small networks. The routing tables are computed centrally, by the gateway node,
using network discovery mechanism described in the following section, and
distributed to each node in the network.

\subsection{Network Discovery and Routing Table Calculation}

Periodically, the gateway node performs a discovery procedure. It
broadcast a discovery request packet. Each node which receives
a discovery packet
\begin{enumerate}
\item takes note where the source node it came from: this is
the next hop to the gateway
\item marks itself as visited
\item responds to the request by sending a response packet
to the node from which the request came
\item broadcasts the same discovery request packet
\end{enumerate}

When a node gets a response packet, it forwards it to the next
hop towards the gateway by using the hop information it saved
in the first step above.

Each discovery procedure is marked by a version sequence number and all of the
above actions and checks apply to one sequence number only.

As the gateway receives the responses, it fills a graph data structure
(adjacency list). It does so by reading the traversed links from the
\texttt{path} field, which is stored in the packet. The \texttt{path} field
contains a trace of the packet's hops over nodes, which is maintained by our
lowest-level packet code.

Once it has the graph, the gateway calculates the shortest paths from each node
to all other nodes using repeated invocations of Dijkstra's algorithm. From the
shortest path information, the gateway node builds the routing tables for all
nodes and distributes them.

\section{Metrics}

\subsection{Low Latency}

Latency is relevant for the virtual fence in order to detect a
fence crossing and notify the safety monitor before the
misbehaving robot goes too far out of the fenced region.

Our design is optimized for low latency thanks to the routing by next hop
tables, which contain the next hop calculating according to the shortest path
to destination by hop count. See Section~\ref{sec:design} for info
on computing the shortest-path routing tables.

\subsection{Reliability}

Reliability is important for a virtual fence because the fence crossing event
delivery must happen even if some intermediate nodes malfunctioned.

Our design is optimized for reliability by having the routed packets
acknowledged and by having nodes update their routing tables when an
acknowledgement does not arrive in time. As a result, if a node X dies, the
nodes which list X as the next hop in their routing tables, will replace it
automatically.

For example, if a node A gets a message to C, and A's routing table says that
the next hop to C is B, then A sends the packet to B, but B is dead. If A does
not get an ack from B for this packet, then A updates its routing table to no
longer list B as the next hop for C. Instead, A picks a neighbor and adds it as
the next hop. On next packet, A will not forward it to B but to a different
neighbor. Currently, the choice of neighbor is random (except for packet
source), but in the future it could be based on last heard time and RSSI, both
of which are tracked.

\end{document}
